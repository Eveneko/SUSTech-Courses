% !TEX TS-program = xelatex
% !TEX encoding = UTF-8
% !Mode:: "TeX:UTF-8"

\documentclass[onecolumn,oneside]{SUSTechHomework}

\usepackage{indentfirst}
\setlength{\parindent}{2em}

\author{Yubin Hu}
\sid{11712121}
\title{Assignment 2}
\coursecode{CS403}
\coursename{Cryptography and Network Security}

\begin{document}
  \maketitle

  \section{}

  $F$ is $not$ a PRF.

  Suppose we have the distinguisher $D$,
  $D$ queries $F_{A,b}(0)$ which equals $b$. $D$ can compute b with one query.
  Because b is known, and we know that $D$ can compute $Ax$ by every $x$.
  $D$ can compute A by posing queries for the unit vectors.
  So $F$ is $not$ a PRF.

  \section{}

  Assume the statement is false, which means there exist some PPT adversary $A$,

  $$|Pr[PrivK_{A,\Pi}^{CPA}(n)=1]-Pr[PrivK_{A,\widetilde{\Pi}}^{CPA}(n)=1]| > t(n)$$

  For some non-negligible $t(n)$.
  We constructs a PPT distinguisher D contradicting the requirement from the definition of pseudo-random functions, so we have:
  
  $$|Pr[D^{F_k(.)}(1^n)]-Pr[D^{f(.)}(1^n)]|>t(n)$$

  We assume the working of the distinguisher $D$, which is given access to some oracle $O:\{0,1\}^n \rightarrow \{0,1\}^n$ and receives an input $1^n$

  \begin{itemize}
    \item Run $A(1^n)$, and query $A(1^n)$'s oracle to the encryption function for $i$-th time with a message made up by $l_i$ message blocks($m_1, m_2,..,m_{l_i}$), we do:
    \begin{itemize}
      \item Choose a uniform initial value $ctr^* \in \{0,1\}^m$
      \item Query $O$ for $j=1,..,l_i$ to obtain $y_j=O(ctr_i+j)$
      \item Return the ciphertext blocks $<ctr_i,c_i,...,c_{l_i}>=<ctr_i, y_i \oplus m_{b,1},...,y_{l*} \oplus m_{b,l^*}>$ to $A$.
    \end{itemize} 
    \item Once A output the messages $m_0$, $m_1$ consisting of $l^*$ blocks $m_{0,1},...,m_{0,l*},m_{1,1},..,m_{1,l*}$ respectively, choose a uniform bit $b \in {0,1}$, we do:
    \begin{itemize}
      \item Choose a uniform initial value $ctr^* \in \{0,1\}^m$
      \item Query $O$ for $j=1,..,l_i$ to obtain $y_j=O(ctr_i+j)$
      \item Return the ciphertext blocks $<ctr_i,c_i,...,c_{l_i}>=<ctr_i, y_i \oplus m_{b,1},...,y_{l*} \oplus m_{b,l^*}>$ to $A$.
    \end{itemize}
    \item Answer queries to the encryption oracle as above, until $A$ produces an
    output bit $b'$. Then output 1 if $b=b'$, and 0 otherwise.
  \end{itemize}

  We argue D is PPT hence D runs in polynomial time.

  Note that $D$ is essentially just the experiment $PrivK_{A,\Pi}^{CPA}$ or $PrivK_{A,\widetilde{}{\Pi}}^{CPA}$ depending on witch oracle D is given.
  with the first step of key generation in the experiment being simulated by uniformly choosing $k \in \{0,1\}^n$ or $f \in Func_n$.

  $$Pr_{k \leftarrow (0,1)^n}[D^{F_k(.)}(1^n)=1]=Pr[PrivK_{A,\Pi}^{CPA}(n)=1]$$
  $$Pr_{k \leftarrow Func_n}[D^{F_k(.)}(1^n)=1]=Pr[PrivK_{A,\widetilde{}{\Pi}}^{CPA}(n)=1]$$

  \section{}

    \subsection{}

    This scheme has indistinguishable encryptions in the presence of an
    eavesdropper.

    We prove that $F_k(0^n)$ is pseudorandom.

    The scheme is not CPA-secure as encryption is deterministic.

    \subsection{}

    The one-time pad is perfectly indistinguishable.

    It has also indistinguishable encryption in the presence of an
    eavesdropper.

    . It is not CPA-secure as encryption is deterministic.

    \subsection{}

    This scheme does not even have indistinguishable encryptions in the
    presence of an eavesdropper.

    It cannot be CPA-secure.

    \subsection{}

    This scheme is CPA-secure.

    It has indistinguishable encryptions in the presence of an eavesdropper 

  \section{}

    \subsection{}

    This is not CPA-secure. Follow this attacker $A$:

    \begin{itemize}
      \item The key $k$ is chosen at random and fixed.
      \item A get 2 different messages $m_0,m_1 \in \{0,1\}^{2m}$. Then A interacts with the encryption oracle $E_k$ and obtains two ciphertexts $y_0, y_1$
      \item A sends $m_0,m_1$ to the challenger and gets $c^*=E_k(m_b)$ for $b \leftarrow _R\{0,1\}$. A intercepts the first $2m$ bits of $c^*$ as $c'$
      \item if $c'$ is identical to the first $2m$ bits of $y_0$, A outputs $b=0$; otherwise, $b=1$
    \end{itemize}

    Therfore, $A$ wins the game with probability 1.

    \subsection{}

    This scheme is CPA-secure.
    Considering $y_1=p_k(0^n \oplus r)$ as a random function.

  \section{}

  We construct an adversary $A$ for each of the MACs.

    \subsection{}

    On input $1^n$, $A$ queries $(0^n1^n)$ and gets $t=Mac_k(0^n1^n)=F_k(0^n) \oplus F_k(1^n)$.
    Now A outputs $(1^n0^n, t)$.
    This is a valid messages-tag pair as $Mac_K(1n0^n)=F_k(1^n) \oplus F_k(0^n)=F_k(0^n) \oplus F_k(1^n)= t$.
    So A wins with probability 1.

    \subsection{}

    On input $1^n$, $A$ queries $m_0=0^n$, $m_1=0^{\frac{n}{2}}1^{\frac{n}{2}}$ and $m_2=1^n$. We donote the tags as $t_0$, $t_1$, $t_2$.

    \begin{equation}
      \begin{split}
      t_0 \oplus t_1 \oplus t_2 &\\
      &= (F_k([1]||0^{\frac{n}{2}}) \oplus F_k([2]||0^{\frac{n}{2}}) \oplus F_k([1]||0^{\frac{n}{2}}) \oplus F_k([2]||1^{\frac{n}{2}}) \oplus F_k([1]||1^{\frac{n}{2}}) \oplus F_k([2]||1^{\frac{n}{2}})) \\
      &= F_k([2]||0^{\frac{n}{2}}) \oplus F_k([1]||1^{\frac{n}{2}}) - F_k([1]||1^{\frac{n}{2}}) \oplus F_k([2]||0^{\frac{n}{2}}) \\
      &= Mac_k(1^{\frac{n}{2}}0^{\frac{n}{2}})
      \end{split}
    \end{equation}

    Therfore, $A$ outputs $(1^{\frac{n}{2}}0^{\frac{n}{2}},t_0 \oplus t_1 \oplus t_2)$ and wins with probability 1.

    \subsection{}

    Let $m \in {0,1}^{\frac{n}{2}}$ be an arbitary messages. Then A outputs $(m,([1]_2||m))$.
    This is a valid message-tag pair as $Mac_k$ could choose $r=[1]_2||m$ and output

    $$t=(r,F_k(r) \oplus F_k([1]_2||m))=(r,0^n)$$

    Therfore, A wins with probability 1.

  \section{}

  For arbitary two messages $m_1, m_2 \in {0,1}^n$ and $m_1, m_2 \neq 0^n$.
  So we can query the oracle $t_1=Mac_k(m_1||0^n), t2=Mac_k(0^n||m_2)$. Then let $c_1$ be the first $\frac{n}{2}$ bits of $t_1$ and $c_2$ be the last $\frac{n}{2}$ bits of $t_2$.
  Then A can output $(m_1||m_2,c_1||c_2)$.

  \section{}

  $\widetilde{H}$ is a collision resistant hash function.
  We prove by reduction.

  Assume that $\widetilde{H}$ is not a collision resistant.
  Then there is a PPT adversary $\widetilde{A}$ such that:

  $$Pr[Hash~-~col_{\widetilde{A},\widetilde{\Pi}}(n)=1] \ge \frac{1}{q(n)}$$

  We use $\widetilde{A}$ to constructs $A$ as follows: On input $s$, $A$ simulates $\widetilde{A}$. The latter will output $x$, $x'$ eventually. Now $A$ checks whether $H^8(x)=H^8(x')$ and $x \neq x'$. If this is not the case A will just output $x$ and $x'$.
  Otherwise, $A$ will checks whether $\widetilde{H}^8(x)=\widetilde{H}^8(x')$.
  If this is the case, then a collision was found and $A$ outputs $x$ and $x'$.
  Otherwise, $H^8(x) \neq H^8(x')$ and $H^8(H^8(x))=H^8(H^8(x'))$, a collision is found too.

  We have:

  $$Pr[Succ_{\widetilde{A}}(n)]=Pr[Hash~-~col_{\widetilde{A},\widetilde{\Pi}}(n)=1]>\frac{1}{q(n)}$$

  $$Pr[Hash~-~col_{\widetilde{A},\widetilde{\Pi}}(n)=1|Succ_{\widetilde{A}}(n)]=1$$

  \begin{equation}
    \begin{split}
      Pr&[Hash~-~col_{\widetilde{A},\widetilde{\Pi}}(n)=1 \\
      & = Pr[Hash~-~col_{\widetilde{A},\widetilde{\Pi}}(n)=1|Succ_{\widetilde{A}}(n)]*Pr[Succ_{\widetilde{A}}(n)] + Pr[Hash~-~col_{\widetilde{A},\widetilde{\Pi}}(n)=1|\neg Succ_{\widetilde{A}}(n)]*Pr[\neg Succ_{\widetilde{A}}(n)] \\
      & \ge Pr[Hash~-~col_{\widetilde{A},\widetilde{\Pi}}(n)=1|Succ_{\widetilde{A}}(n)]*Pr[Succ_{\widetilde{A}}(n)] \\
      & = 1*Pr[Succ_{\widetilde{A}}(n)] \\
      & > \frac{1}{q(n)}
    \end{split}
  \end{equation}

  This completes the reduction as $\Pi$ is a collision resistant hash function.
Therefore our assumption was wrong and $\widetilde{\Pi}$ is collision resistant.

  \section{}



  \section{}

    \subsection{}

    Let $k=0^n$ and $m \in \{0,1\}^n$ be arbitary.
    Then $f_1(0^n, m)=E(0^n,m)$.
    Let $k' \in {0,1}^n$ also be arbitary where $k' \neg k$.
    Then let $m'=E^{-1}(k',E_{0^n}(m) \oplus k') \oplus k'$
    So $f_1(0^n, m) = f_1(k', m')$

    \subsection{}

    \subsection{}

    the same as 9.1

  \section{}

    \subsection{}

    The probability that both of them receive the same plate number is $\frac{1}{10 \time 10 \time 26}= \frac{1}{2600}$

    \subsection{}

    $1 \times (1 - \frac{1}{2600}) \times (1 - \frac{2}{2600}) \times ... \times (1 - \frac{n}{2600}) > 1\%$

    So $n=6$, the maximum number of this type of license plates is $n+1=7$.

    \subsection{}

    $1 \times (1 - \frac{1}{2600 \times 10^m}) \times (1 - \frac{2}{2600 \times 10^m}) \times ... \times (1 - \frac{49}{2600 \times 10^m}) > 1\%$

    So $m=2$, which is the digits should be added at the end of this serial number format.

\end{document}
