% !TEX TS-program = xelatex
% !TEX encoding = UTF-8
% !Mode:: "TeX:UTF-8"

\documentclass[onecolumn,oneside]{SUSTechHomework}

\usepackage{indentfirst}
\setlength{\parindent}{2em}

\author{Yubin Hu}
\sid{11712121}
\title{Assignment 1}
\coursecode{CS403}
\coursename{Cryptography and Network Security}

\begin{document}
  \maketitle

  \section*{Q.1}
  Due to the shift-cipher's key is from 0 to 25, so we try all 26 keys. When the $key = 13$, we decrypt the cihertext:

  "SUSTech is a public university founded in the lush hills of Nanshan District Shenzhen It is working towards becoming a world class university excelling in interdisciplinary research nurturing innovative talents and delivering new knowledge to the world."

  \section*{Q.2}
  We prove that for every pair of plaintexts $x, x'$, the probability that Eve guesses $x_b$ after seeing the ciphertext $c=Enc_k(x_b)$ is at most 1/2, if and only if we have $E_{U_n}(x) \equiv E_{U_n}(x')$.

  \subsection*{"if" part}
  if $E_{U_n}(x) \equiv E_{U_n}(x')$, the Eve can not get any information and the probability is at most.

  \subsection*{"only if" part}
  We prove by contrapositive.
  Suppose $\exists x_1, x_2$ such that $E_{U_n}(x_1) \not\equiv​ E_{U_n}(x_2)$ which means that there has a $y_0$ satisfied that $Pr[Y_{x_1}=y_0]>Pr[Y_{x_2}=y_0]$. So that Eve guesses $x_1$ if the ciphertext $y=y_0$, otherwise Eve guesses randomly. Then the probability is larger than 1/2.

  \section*{Q.3}

  \subsection*{"if" part}
  Suppose that $Pr[E_{nc_k}(m)=c]=Pr[E_{nc_k}(m')=c]$, we should prove that (Gen,Enc,Dec) with message space $M$ is perfectly secure, which means:
  $$Pr[M=m|C=c]=\frac{Pr[C=c|M=m]*Pr[M=m]}{\sum_{m' \in M}Pr[C=c|M=m']*Pr[M=m']}=Pr[M=m]$$
  $$Pr[C=c|M=m]*Pr[M=m]=\sum_{m' \in M}Pr[C=c|M=m']*Pr[M=m']$$
  Since $\forall m, m' \in M$, we have $Pr[C=c|M=m]=Pr[C=c|M=m']$, so that
  $$\sum_{m' \in M}Pr[C=c|M=m']*Pr[M=m']=Pr[C=c|M=m']*\sum_{m' \in M}Pr[M=m']=Pr[C=c|M=m]$$
  Above all, we complete the proof.


  \subsection*{"only if" part}
  Suppose that (Gen,Enc,Dec) with message space $M$ is perfectly secure. Fix arbitrary messages $m, m'$ and ciphertext $c$, since it is perfectly secure,
  $$\forall m \in M,~Pr[M=m|C=c]=Pr[M=m]$$
  So we have
  $$Pr[E_{nc_k}(m)=c]=\frac{Pr[M=m|C=c]*Pr[C=c]}{Pr[M=m]}=Pr[C=c]$$
  $$Pr[E_{nc_k}(m')=c]=\frac{Pr[M=m'|C=c]*Pr[C=c]}{Pr[M=m']}=Pr[C=c]$$
  The resuly is nothing to do with $m$, $Pr[E_{nc_k}(m)=c]=Pr[E_{nc_k}(m')=c]$

  \section*{Q.4}
  
  \subsection*{(1)}
  Yes. For every $m \in Z_M$, we have $Pr[E_k(m)=c]=\frac{|{k \in Z_M~:~k~+~m~mod~M~=~c}|}{|Z_M|} = \frac{1}{6}$. Due to an encryption scheme (Gen,Enc,Dec) with message space $M$ is perfectly secure if and only if $Pr[E_{nc_k}(m)=c]=Pr[E_{nc_k}(m')=c]$, this encryption scheme is \emph{perfectly secure}.

  \subsection*{(2)}
  No. Let $m_1=1,~m_2=0,~c=4$, $Pr[E_{nc_k}(m_1)=c]=0$, $Pr[E_{nc_k}(m_2)=c]=\frac{1}{3}$, they go against \emph{an encryption scheme (Gen,Enc,Dec) with message space $M$ is perfectly secure if and only if $Pr[E_{nc_k}(m)=c]=Pr[E_{nc_k}(m')=c]$}, so this encryption scheme is not perfectly secure.

  \section*{Q.5}
  for any two messages $m_0,m_1$, $m_0$ and $m_1$ agree on at least $l/2$ bits.. Suppose there is a new message $m_2$, $m_2$ and $m_1$ also agree on at least $l/2$ bits.

  $m_2$ could be composed of the first half of $m_0$ and the second half of $m_1$. So that we have $E_{U_n}(m_0) \equiv E_{U_n}(m_1) \equiv E_{U_n}(m_2)$ which prove that any encryption scheme that is half-message perfectly secure must in fact also be perfectly secure.

  \section*{Q.6}
  To prove that the statistical distance $\triangle(X, Y)$ defined in Definition 2.1 of Lecture 3 is a metric. We should prove:
  
  (1) $\triangle(X, Y) = 0 \Leftrightarrow (X, Y)$

  since
  $$\triangle(X, Y) = max_{T\subseteq {0,1}^n}$$
  $$|Pr[X \in T]-Pr[Y |in T]| = 0$$
  which means $t \in {0,1}^n$ and $Pr[X \in T] = Pr[Y |in T]$.
  Suppose $\forall t \in {0,1}^n$
  $$Pr[X=t]=Pr[X \in {t}]=Pr[X \in {t}]=Pr[Y \in {t}]=Pr[Y=t]$$
  So that $X=Y$
  
  (2) $\triangle(X, Y) = \triangle(Y, X)$

  since
  $$\triangle(X, Y) = max_{T\subseteq {0,1}^n}$$
  so that
  $$|Pr[X \in T]-Pr[Y |in T]| = max_{T\subseteq {0,1}^n}$$
  $$|Pr[Y \in T]-Pr[X |in T]| = \triangle(Y, X)$$
  
  (3) $\triangle(X, Y) \le \triangle(X, Z) + \triangle(Z, Y)$

  $\triangle(X, Y) = \frac{1}{2}\sum_{w \in Supp(X) \cup Supp(Y)}|Pr[X=w]-Pr[Y=w]|$ So that:
  \begin{align*}
    \triangle(X, Z) &+ \triangle(Z, Y) \\
   &= \frac{1}{2}\sum_{w \in Supp(X) \cup Supp(Y)}|Pr[X=w]-Pr[Z=w]| + \frac{1}{2}\sum_{w \in Supp(X) \cup Supp(Y)}|Pr[Z=w]-Pr[Y=w]|  \\
   &= \frac{1}{2}\sum_{w \in Supp(X) \cup Supp(Y)}(|Pr[X=w]-Pr[Z=w]| + |Pr[Z=w]-Pr[Y=w]|)   \\
   &\ge \frac{1}{2}\sum_{w \in Supp(X) \cup Supp(Y)}(|Pr[X=w]-Pr[Z=w] + Pr[Z=w]-Pr[Y=w]|)   \\
   &= \frac{1}{2}\sum_{w \in Supp(X) \cup Supp(Y)}(|Pr[X=w]-Pr[Y=w]|)   \\
   &= \triangle(X, Y)
  \end{align*}

  So the statistical distance $\triangle(X, Y)$ defined in Definition 2.1 of Lecture 3 is a metric.



  \section*{Q.7}
  To prove that the computational indistinguishability $\approx$ defined above is an equivalence relation, we need to prove that $\approx$ is relexive, symmetric and transitive. 
  \subsection*{reflexive}
  $X_n \approx Y_n$ for any $\epsilon$, $Pr[A(X_n)=1]-Pr[A(Y_n)=1] \le \epsilon(n)$, whith means $X_n \approx X_n$
  \subsection*{symmetric}
  If $X_n \approx Y_n$, $|Pr[A(X_n)=1]-Pr[A(Y_n)=1]|=|Pr[A(Y_n)=1]-Pr[A(X_n)=1]|$, whith means $X_n \approx X_n$
  \subsection*{transitive}
  If $X_n \approx Y_n$ and $Y_n \approx Z_n$, for any $\epsilon$, $Pr[A(X_n)=1]-Pr[A(Y_n)=1] \le \epsilon(n)$, $Pr[A(Y_n)=1]-Pr[A(Z_n)=1] \le \epsilon(n)$

  \begin{align*}
    Pr&[A(X_n)=1]-Pr[A(Y_n)=1] \\
   &= Pr[A(X_n)=1]-Pr[A(Y_n)=1] + Pr[A(Y_n)=1]-Pr[A(Z_n)=1]  \\
   &\le |Pr[A(X_n)=1]-Pr[A(Y_n)=1]| + |Pr[A(Y_n)=1]-Pr[A(Z_n)=1]|  \\
   &\le 2\epsilon(n)
  \end{align*}


\end{document}
