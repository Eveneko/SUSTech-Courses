% !TEX TS-program = xelatex
% !TEX encoding = UTF-8
% !Mode:: "TeX:UTF-8"

\documentclass[onecolumn,oneside]{SUSTechHomework}

\usepackage{indentfirst}
\setlength{\parindent}{2em}

\author{Yubin Hu}
\sid{11712121}
\title{Assignment 2}
\coursecode{CS403}
\coursename{Cryptography and Network Security}

\begin{document}
  \maketitle

  \section{}

  Since $X_n \approx Y_n$, then for any polynomial-time algorithm $A$, we have:
  $$|Pr[A(X_n)=1]-Pr[A(Y_n)=1]| \le \epsilon(n)$$
  For any polynomial-time algorithm A, we also have another algorithm $A'$ since $A'(x) = A(f(x))$, sice f is a polynomial-time computable function, so:
  $$|Pr[A'(X_n)=1]-Pr[A'(Y_n)=1]| \le \epsilon(n)$$
  $$|Pr[A(f(X_n))=1]-Pr[A(f(Y_n))=1]| \le \epsilon(n)$$
  therefore this means we have:
  $$f(X_n) \approx f(Y_n)$$

  \section{}

    \subsection{}

    We prove it by contraposition.
    Suppose the negligible function $negl_3$ is not negligible, which means:
    $$\forall n,~\exists p,~negl_3(n) \ge \frac{1}{p(n)}$$
    which $p$ is a polynomial, then,
    $$negl_1(n) \ge \frac{1}{2p(n)}~\mbox{or}~negl_2(n) \ge \frac{1}{2p(n)}$$
    We assume that $negl_1(n) \ge \frac{1}{2p(n)}$, then assume polynomial $p'=2p$:
    $$negl_1(n) \ge \frac{1}{p‘(n)}$$
    which means $negl_1$ is not negligible function.
    Therefore it is contraposition, $negl_3$ is a negligible function.

    \subsection{}

    We prove it by contraposition.
    Suppose the negligible function $negl_3$ is not negligible, which means:
    $$\forall n,~\exists q,~negl_4(n) \ge \frac{1}{q(n)}$$
    which $p$, $q$ is a polynomial, then:
    $$negl_1(n) \ge \frac{1}{p(n) \times q(n)}$$
    then assume polynomial $q'(n) = p(n) \times q(n)$, then we have
    $$negl_1(n) \ge \frac{1}{q‘(n)}$$
    which means $negl_1$ is not negligible function.
    Therefore it is contraposition, $negl_4$ is a negligible function.

  \section{}

  Suppose $p_{i,j} = Pr[A(E_{U_m}(x_i))=j]$, we get $p_{i,0}+p_{i,1}+p_{i,2}=1$. For each $i = 0,1,2$, we have $|p_{i,j}-p_{k,j}| \le \epsilon(n)$, where $i,j,k \in {0,1,2}$, then $p_{k,j} \ge p_{j,i} - \epsilon~\mbox{for}~k \neq j$.
  So $3 = \sum_{i=0}^{2}\sum_{j=0}^{2}P_{ij} \ge 3\sum_{i=0}^{2}p_{i,i}-3\epsilon (n)$
  Thus, $\frac{1}{3}\sum_{i=0}^{2}p_{i,i} \le frac{1}{3}+frac{1}{3}\epsilon (n) < 0.34$

  \section{}

    \subsection{} No. We have that $|Pr_{x \leftarrow X_n}[A(x)=1]| - Pr_{x \leftarrow U_n}[A(x)=1] \le \epsilon (n)$  for every PPT distinguaisher A.
    However, there is a distinguaisher A, that output 1 when $x_n = x_1 \oplus x_2 \oplus ... \oplus x_n-1$. $Pr_{x \leftarrow X_n}[A(x)=1]$ = 1 and $Pr_{x \leftarrow U_n}[A(x)=1] = \frac{1}{2}$, so $X_n$ is not pseudorandom.

    \subsection{} Yes. With $1-2^{-n/10}$ probability $Z_n$ have a random string, so we have $|Pr_{x \leftarrow Z_n}[A(x)=1] - Pr_{x \leftarrow U_n}[A(x)=1]| \le |2^{-n/10} - 2^{-n/10} \times Pr_{x \leftarrow U_n}[A(x)=1]| \le 2^{-n/10}$. Since $2^{-n/10}$ is a negligible function, $Z_n$ is pseudorandom.

  \section{}

    \subsection{}

    Prove 2 first.
    Not necessary. We can prove $G'' = G(s_{n/2+1}...s_n)$ is a PRG. Assume $G'$ is a PRG, we have $G'(s) = G''(s_0^{|s|})=G(0^{|s|})$ is a PRG. By contradicted, it is not necessary.

    \subsection{}

    Yes. We Suppose $l, l'$ are the expansion factor of $G, G'$. Therefore $l(n) = |G(s)|$ and $l'(n) = |G(s_1...s_n{n/2})| = l(\frac{n}{2})$, for any PPT distinguaisher A we have:
    $$|Pr_{x \leftarrow U_n}[A()G'(x)=1] - Pr_{y \leftarrow U_{l'(n)}}[A(y)=1]| \le \epsilon (\frac{n}{2})$$

    Then we prove $\epsilon '(n) = \epsilon(\frac{n}{2})$. We prove it by contraposition.
    Therefore, $\epsilon '$ is a negligible function and $G'$ is a pseudorandom generator.

  \section{}

  $F_k$ is not a PRF.
  We construct a PPT distinguaisher, there are two arbitrary string $x_0$ and $x_1$ and $|x_0| = |x_1| = n$. Output 1 when $f(x_0) \oplus f(x_1) = x_0 \oplus x_1$ and outputs 0 otherwise.
  So the posibility for outputing 1 is $2^{-n}$ if $f$ is a pseudorandom function.
  there is 1 when $f=F_k, k \leftarrow {0,1}^n$, so $|1-2^{-n}|$ is not a negligible function and $F_k$ is not a PRF.

  \section{}

  We prove it by contraposition. If $G$ is not a PRG, which means there is a PPT distinguaisher D and a polynomial p.
  $$|Pr_{k \leftarrow {0,1}^n}[D(F_k(<1>)|F_k(<2>))|...|F_k(<l>)=1] - Pr_{y \leftarrow (0,1)^{ln}}[D(y)=1]| \ge p(n)$$
  Then by the definition we have:
  $$Pr_{y \leftarrow (0,1)^{ln}}[D(y)=1] = Pr_{y \leftarrow (0,1)^{ln}}[D(f(<1>)|f(<2>))|...|f(<l>)=1]$$
  Therefore, there is an PPT distinguaisher $D'$ for given function $f'$ simply output the result of $D(f'(<1>)|f'(<2>))|...|f'(<l>)$. It is contradicted to distinguaish PRF $F_k$ with the random function $f$.
  So G is a PRG.

  \section{}

  Suppose that $IV$ is a n-bit string, we can construct the adversary A. When query the encryption oracle with $m=0^{n-1}1$ and we get the ciphertext $<IV, c>$. If $IV$ is odd(last bit 1), output a random bit; If $IV$ is even(last bit 0), output $m_0 = 0^n$ and arbitrary $m_1$ to be encrypted. Then  Receive the challenge ciphertext $<IV+1, c'>$, and output 0 if $c'=c$, and 1 otherwise.

  We claim that this adversary succeeds with probability that is greater than $\frac{1}{2}$ by a nonnegligible function. By guessin randomly, A succeeds with probability $\frac{1}{2}$ if $IV$ is odd, which is $\frac{1}{4}$. If $IV$ is even, $IV+1=IV \oplus 0^{n-1}$. Therefore, $c=F_k(IV \oplus m_0)=F_k(IV \oplus 0^{n-1}1)=F_k(IV+1)=F_K(IV+1 \oplus 0)=F_K(IV+1 \oplus m_0)$. If $m_0$ is encrypted, then $c=c'$. That is, whenever IVis even, A decides correctly which message was encrypted. This is $\frac{1}{2}$ cases. In total, A wins $\frac{3}{4}$ cases.


\end{document}
