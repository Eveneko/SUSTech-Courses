% !TEX TS-program = xelatex
% !TEX encoding = UTF-8
% !Mode:: "TeX:UTF-8"

\documentclass[onecolumn,oneside]{SUSTechHomework}

\usepackage{indentfirst}
\setlength{\parindent}{2em}

\author{Yubin Hu}
\sid{11712121}
\title{Assignment 4}
\coursecode{CS403}
\coursename{Cryptography and Network Security}

\begin{document}
  \maketitle

  \section{}
  
    \subsection{}
    
    $H_a$ is necessarily collison-resistant.

    Here we use the contradiction method to prove.
    Suppose $H_a$ is not necessarily collison-resistant, there is a PPT algorithm A which find a collison with non-negligible probability($x \neq y,~H_a(x) = H_a(y)$).
    So we have $H_1^{s_1}(x)||H_2^{s_2}(x)=H_1^{s_1}(y)||H_2^{s_2}(y)$, and we got $H_1(x) = H_1(y)$ and $H_2(x) = H_2(y)$, which means $H_1 \mbox{and} H_2$ are not collison-resistant.
    This conflicts with assumptions. 

    \subsection{}

    $H_b$ is not necessarily collison-resistant.

    Suppose we have $H_1$ be collison-resistant and $H_2(x) = 0$ for all $x$.
    So we got $\forall x, H_b(x) = H_1(0)||0$, and in this case, $H_b$ is not necessarily collison-resistant.

    \subsection{}

    $H_c$ is necessarily collison-resistant. 

    $H_c$ is necessarily collison-resistant.
    Suppose $H_c$ is not necessarily collison-resistant, there is a PPT algorithm A which find a collison with non-negligible probability($x \neq y,~H_c(x) = H_c(y)$).
    So we have $H_1(H_2(x)||x)||H_2(H_1(x)||x) = H_1(H_2(y)||y)||H_2(H_1(y)||y)$, and we got $H_1(H_2(x)||x) = H_1(H_2(y)||y)$ and $H_2(H_1(x)||x) = H_2(H_1(y)||y)$.
    Also we know that $x \neq y$, so $H_1(x)||x \neq H_1(y)||y$ and $H_2(x)||x \neq H_2(y)||y$, , which means $H_1 \mbox{and} H_2$ are not collison-resistant.
    This conflicts with assumptions. 

  \section{}

    \subsection{}
    
    Suppose that we have the compression function $h: {0,1}^{2n} \rightarrow {0, 1}^n$ and the length of $k$ is $n$.
    
    1. An arbitary message $m$ with length $n$ to ask the oracle. Let $t = Mac(m) = H(k||m) = h(h(k||IV)||m)$.

    2. $Mac(m||t) = h()t||t$, since $Mac(m||t)=H(k||m||t) = h(h(h(k||IV)||m)||t)$ and $t = h(h(k||IV))$

    Thus this PPT algorithm has a winning probability 1.
    This is not a secure MAC.

    \subsection{}

    If $H$ is modeled as a random oracle, $F_k(M) =H(k||m) is PRF$.
    And it is MAC security.

    \subsection{}

    The random oracles are more soundness then the normal oracles.
    So the consequences are only available for the random oracles, not for all the cases.

\section{}

  \subsection{}
  
  There are the set of quadratic residue (QR) called $G$. $G \subseteq Z_n^*$.
  Suppose we have $y_1,y_2 \in G$, then $\exists x_1,x_2 \in Z_n^*$, $y_1 = x_1^2,y_2=x_2^2$.
  So $y_1y_2 = x_1^2x_2^2 = x_1x_2x_1x_2$, it is closure.
  And $1=1^2$, $1 \in G$, it is identity.
  Now we have $y \in G$, $\exists x,x^{-1} in G_n^*$, so that $y=x^2,xx^{-1}=1$.
  $y^{-1} = {x^2}^{-1} = (x^{-1})^2$ and $y^{-1} \in G$, $yy^{-1} = 1$.
  Therefore, the set of QRs is a subgroup of $Z_n^*$.

  \subsection{}

  \begin{itemize}
    \item \emph{if part} Suppose we have $y \in Z_p^*$, $log_g(y) = 2k$, then $g^{2k} = y, y = (g^k)^2$ and $y$ is a QR. 
    \item \emph{only if part} Support that $y$ is a QR, then $\exists x \in Z_p^*$ such that $y = x^2$. Let $log_g(x) = i$, then we got $x = g^i$ and $y = g^{2i}$.
    \item $log_g(y)$ is the min number over all the $2i \mbox{mod} (p-1)$.
    \item $p$ is a prime, $p - 1 = 1$ or $p$ is a even number. So $log_g(y)$ is even number. 
  \end{itemize}

\section{}

Suppose we have $Z_n$ has a generator $g$.
$gcd(g, N) = 1$ and for $\forall y \in Z_N$.
We know that in $Z_N$, $g^x = xg \mbox{mod} N$.
Thus $x = g^{-1}y$ and We can obtain x through the extended Euclidean algorithm.

\section{}

  \subsection{}

  $S={1,4,9,16,8,2,15,13,13,15,2,8,16,9,4,1}={1,2,4,8,9,13,15,16}$, the size is 8.

  \subsection{}

  $x \in Z_{17}^*$ is a generator if and only if $gcd(x, \varphi(17))=1$. So the number of generator is $\varphi(\varphi(17))=\varphi(16)=8$

  \subsection{}

  $g^{ab} \in S$,
  $ab \mbox{mod} 2 = 0$, 17 is prime.
  $Pr[g^{ab} \in S]=1-Pr[ab~is~odd]=1-\frac{1}{4}=\frac{3}{4}$
  
\section{}

Suppose $x$ is a random element of $Z_N^*$ and $y=x^2$, we have a algorithm $A$ to get $y's$ square root $z$.
if $z = \pm x$, we choose another $x$ and do it again, until $z \neq \pm x$.
So we got 5 square root of $y: \pm x \pm z$.

$a = x^2 (mod N) = z^2 (mod N)$, $x^2-z^2=(x+z)(x-z) = kN$, $x+z \neq 0$, $x-z \neq 0$, then $k \neq 0$
So $k=1$ and $N=(x+z)(x-z)$

\section{}

For arbitrary message $m$, we first choose an arbitrary $k \neq 0,1$.
Asking the signing oracle to sign $m' = mk^e$ mod $N$(here $e$ is public key in the scheme of signature).
Then we have the $Sign(m') = m^{fd}$ mod $N=(mk^e)^d$ mod $N=m^d \times k^{ed}$ mod $N=m^d$ mod N.
Therefore, message m cannot be queried to the signing oracle.

\section{}

Suppose $G$ is a group with generator $g$, and $h_1, h_2, h_3$ are the element of G.
$h_1 = g^x$, $h_2=g^y$.
We define $DH_h(h_1,h_2)=DH_g(g^x, g^y)=g^{xy}$
The discrete algorithm problem is to compute $log_gh$
The CDH problem is to compute $DH_g(h_1,h_2)$
The DDH problem is distinguish $DH_g(h_1,h_2)$ frome a uniform element of $G$.
DDH > CDH > DLog

\section{}

  \subsection{}
  
  EI Gamal encryption scheme is not secure against the chosen ciphertext atatch.
  CCA-secure schemes are not malleable.

  \subsection{}
  The El Gamal signature scheme using hash-then-sign paradigm is secure against the chosen plaintext attack.

  \subsection{}
  Yes. we forge a signature for any given message m by asking the signing oracle.

\end{document}
